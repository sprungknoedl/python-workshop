\documentclass{beamer}
\usepackage[utf8]{inputenc}
\usepackage[ngerman]{babel}
\usepackage{graphicx}
\usepackage{lmodern}
\usepackage{listings}
\usepackage{color}

\lstset{
    language=Python,
    basicstyle=\scriptsize,
    breakatwhitespace=true,breaklines=true,%
    tabsize=2, % Tabulatorbreite
    showstringspaces=false,
}

\hypersetup{
    pdftitle={\inserttitle},
    pdfsubject={\inserttitle},
    pdfauthor={\insertauthor},
    pdfkeywords={\inserttitle},
    pdfstartview={FitH}
}

\usetheme{default}
\beamertemplatenavigationsymbolsempty
\titlegraphic{\includegraphics[height=2cm]{hk-logo.png}}
%\logo{\includegraphics[height=1cm]{hk-logo.png}}

\title[Python Workshop]{Hackinggroup -- Python Workshop}
\subtitle{A tale about dutch ducks with a fable for British comedy}
\author[Thomas Kastner, Michael Rodler]{Thomas Kastner\and Michael Rodler}
\date{\today}

\begin{document}

\begin{frame}
    \titlepage
\end{frame}

\begin{frame}
    \frametitle{Table of contents -- Wos ma heit mochn}
    \tableofcontents
\end{frame}

\section{Introduction}
%%%%%%%%%%%%%%%%%%%%%%

\begin{frame}
    \frametitle{Introduction -- Wer san ma denn?}

    \begin{block}{Michael Rodler}
    \begin{itemize}
        \item aka f0rk, f0rki, f0rkmaster, Gabel, etc.
        \item Student SIB09
        \item 3 years coding python for fun
        \item 3 months coding python for profit
    \end{itemize}
    \end{block}

    \begin{block}{Thomas Kastner}
    \begin{itemize}
        \item aka br3z3l, tom
        \item Student SIB08
        \item 4 years coding python for fun
    \end{itemize}
    \end{block}
\end{frame}


\begin{frame}[fragile]
	\frametitle{GO! -- TU ES!}
	\pause
    \begin{exampleblock}{python}
    \begin{lstlisting}
>>> import antigravity
    \end{lstlisting}
    \end{exampleblock}
\end{frame}


\section{Syntax}
%%%%%%%%%%%%%%%%
\subsection{Hello World}
\begin{frame}[fragile]
    \frametitle{Every tutorial has this -- Braucht ma oafoch}
    \begin{block}{Wikipedia about ``Hello World!''}
    A ``Hello World'' program is a computer program which prints out ``Hello World!'' on a display device. It is used in many introductory tutorials for teaching a programming language.
    \end{block}

    \begin{exampleblock}{python}
    \begin{lstlisting}
>>> print "Hello World!" 
    \end{lstlisting}
    \end{exampleblock}
\end{frame}

\subsection{About variables \& types}
\begin{frame}[fragile]
	\frametitle{About variables \& types}
	\begin{alertblock}{dynamic strong typed}
	\begin{itemize}
	\item dynamic -- any variable, any type
	\item strong -- no magic type casting
	\end{itemize}
	\end{alertblock}
	\begin{exampleblock}{python}
	\begin{lstlisting}
>>> x = 42
>>> x = "If it looks like a duck and quacks like a duck, it must be a duck."
>>> x = ["beer", "wine", "cheese"]
	\end{lstlisting}
	\end{exampleblock}

	\pause

	\begin{exampleblock}{python}
	\begin{lstlisting}
>>> x = "duck" + 42
TypeError: cannot concatenate 'str' and 'int' objects

>>> x = "duck" + str(42)
	\end{lstlisting}
	\end{exampleblock}
\end{frame}

\begin{frame}[fragile]
	\frametitle{list, dict, tuple -- Eh ois des söbe?}
	
	%TODO: insert pauses between code lines
	\begin{exampleblock}{python}
	\begin{lstlisting}
>>> party = ["cheese", "wine"]
>>> party.append("girls")
>>> party[0] = "beer"
>>> party += ["schnops", "punsch"]
>>> xmasparty = x[2:]
>>> print xmasparty
	\end{lstlisting}
	\end{exampleblock}
	
	\pause
	
	\begin{exampleblock}{python}
	\begin{lstlisting}
>>> x = {"awesome": "barney", 42: "the answer"}
>>> x["awesome"]
>>> x[42]
	\end{lstlisting}
	\end{exampleblock}

	\pause

	\begin{exampleblock}{python}
	\begin{lstlisting}
>>> x = (13, 37)
>>> a, b = x
>>> b, a = a, b
	\end{lstlisting}
	\end{exampleblock}
	%LIVE:
	% Unterschied Listen/Tupel Immutable vs. Mutable
	% Zuweisung an Tupel Element x[1] = "no way!"
	% tupel als dict key {(1,2): "fu"}
\end{frame}


\subsection{Control statements}
\begin{frame}[fragile]
    \frametitle{what \textbf{if}? -- A if-Schleife}
    
    \begin{exampleblock}{python}
    \begin{lstlisting}
>>> if (True or False):
...     print "win"
... else:
...     print "fail"
...
    \end{lstlisting}
    \end{exampleblock}
    
	\pause

    \begin{block}{Truth value testing}
    Any object can be tested for truth value. The following values are considered \emph{false}:
    \begin{itemize}
        \item None, False, 0
        \item any empty sequence, for example: "", (), []
        \item any empty mapping, for example: \{\}
    \end{itemize}
    All other values are considered \emph{true}
    \end{block}
\end{frame}

\begin{frame}[fragile]
    \frametitle{\textbf{for} as long as I live ...}
    
    \begin{exampleblock}{python}
    \begin{lstlisting}
>>> for word in ["python", "is", "awesome"]:
...     print word
    \end{lstlisting}
    \end{exampleblock}

    \begin{exampleblock}{python}
    \begin{lstlisting}
>>> for word in "python is so fucking awesome".split():
...     print word

>>> for character in "python is so fucking awesome":
...     print character
    \end{lstlisting}
    \end{exampleblock}

	\pause

    \begin{block}{Iteration}
    \begin{itemize}
    \item returns next element, each round
    \item every python container type
    \item yo mama's objects %(must implement the python iterator protocol)
    \end{itemize}
    \end{block}
\end{frame}

\begin{frame}[fragile]
	\frametitle{Curiosity killed the cat, but for a \textbf{while} I was a suspect.} %Steven Wright quote (comedian)

	\begin{exampleblock}{python}
	\begin{lstlisting}
>>> while not False:
...     print "print"
	\end{lstlisting}
	\end{exampleblock}
	
	\pause
	
	\begin{exampleblock}{python}
	\begin{lstlisting}
>>> while state != "legendary":
...     wait_for_it()
	\end{lstlisting}
	\end{exampleblock}

	\pause
	
	\begin{block}{jumping}
	you can also \textbf{break} and \textbf{continue}.
	\end{block}
\end{frame}

\subsection{About functions \& methods}
\begin{frame}[fragile]
	\frametitle{About functions \& methods}
	Where is the fucking difference? \pause -- There is none
	\begin{exampleblock}
	
	\end{exampleblock}

\end{frame}

\section{That's why Python is awesome?}
%%%%%%%%%%%%%%%%%%%%%%%%%%%%%%%%%%%
\subsection{Coding style}

\subsection{Code like a pythonista} 
    % Beispiele aus der "Code like a Pythonista" praesentation

\subsection{ALLES ist ein Objekt}


\subsection{Zen of Python} 
\begin{frame}[fragile, shrink]
    \frametitle{Zen of Python}

    \begin{exampleblock}{python}
    \begin{lstlisting}
>>> import this
    \end{lstlisting}
    \end{exampleblock}

    \begin{block}{The Zen of Python, by Tim Peters}
    \scriptsize
    \emph{
    Beautiful is better than ugly.\\
    Explicit is better than implicit.\\
    Simple is better than complex.\\
    Complex is better than complicated.\\
    Flat is better than nested.\\
    Sparse is better than dense.\\
    Readability counts.\\
    Special cases aren't special enough to break the rules.\\
    Although practicality beats purity.\\
    ...}
    \end{block}

\end{frame}
    % import this

\section{Standard Library}
%%%%%%%%%%%%%%%%%%%%%%%%%%
\subsection{Importing modules}
\subsection{sys} 
    % parameter einlesen, basic informationen auslesen
\subsection{os, os.path} 
    % path handling, 
\subsection{RTFM -- Hilfe zur Selbsthilfe}  
    %  help(), pydoc, online doku 

\section{Input/Output } 
    % auf Dateien, stdin/out/err

% Beispiel
\section{re Modul}
% Beispiel

\end{document}
