\documentclass{beamer}
\usepackage[utf8]{inputenc}
\usepackage{graphicx}
\usepackage{lmodern}
\usepackage{listings}
\usepackage{color}

\lstset{
    language=Python,
    basicstyle=\scriptsize,
    breakatwhitespace=true,breaklines=true,%
    tabsize=2, % Tabulatorbreite
}

\hypersetup{
    pdftitle={\inserttitle},
    pdfsubject={\inserttitle},
    pdfauthor={\insertauthor},
    pdfkeywords={\inserttitle},
    pdfstartview={FitH}
}

\usetheme{default}
\beamertemplatenavigationsymbolsempty
\titlegraphic{\includegraphics[height=2cm]{hk-logo.png}}
\logo{\includegraphics[height=0.8cm]{hk-logo.png}}

\title[Python Workshop]{Hackinggroup -- Python Workshop}
\subtitle{A tale about dutch ducks with a fable for British comedy}
\author[Thomas Kastner, Michael Rodler]{Thomas Kastner\and Michael Rodler}
\date{\today}

\begin{document}

\begin{frame}
    \titlepage
\end{frame}

\begin{frame}
    \frametitle{Table of contents -- Wos ma heit mochn}
    \tableofcontents
\end{frame}

\section{Introduction}
%%%%%%%%%%%%%%%%%%%%%%

\begin{frame}
    \frametitle{Introduction -- Wer san ma denn?}

    \begin{block}{Michael Rodler}
    \begin{itemize}
        \item aka f0rk, f0rki, f0rkmaster, Gabel, etc.
        \item Student SIB09
        \item 3 years coding python for fun
        \item 3 months coding python for profit
    \end{itemize}
    \end{block}

    \begin{block}{Thomas Kastner}
    \begin{itemize}
        \item aka br3z3l, tom
        \item Student SIB08
        \item 4 years coding python for fun
    \end{itemize}
    \end{block}
\end{frame}


\section{Syntax}
%%%%%%%%%%%%%%%%
\subsection{Hello World}
\begin{frame}[fragile]
    \frametitle{Every tutorial has this -- Braucht ma oafoch}
    \begin{block}{Wikipedia about "Hello World!"}
    A "Hello World" program is a computer program which prints out "Hello World!" on a display device. It is used in many introductory tutorials for teaching a programming language.
    \end{block}

    \begin{exampleblock}{python}
    \begin{lstlisting}
>>> print "Hello World!" 
    \end{lstlisting}
    \end{exampleblock}
\end{frame}

\subsection{About variables \& types}

\subsection{Control statements}
\begin{frame}[fragile]
    \frametitle{what if?}
    
    \begin{exampleblock}{python}
    \begin{lstlisting}
>>> if (True or False):
...     print "win"
... else:
...     print "fail"
...
    \end{lstlisting}
    \end{exampleblock}

    \begin{block}{Truth value testing}
    Any object can be tested for truth value. The following values are considered \emph{false}:
    \begin{itemize}
        \item None, False, 0
        \item any empty sequence, for example: "", (), []
        \item any empty mapping, for example: \{\}
    \end{itemize}
    All other values are considered \emph{true}
    \end{block}
\end{frame}

\subsection{About functions \& methods}

\section{Warum Python awesome ist!}
%%%%%%%%%%%%%%%%%%%%%%%%%%%%%%%%%%%
\subsection{Coding style}
\subsection{Code like a pythonista} 
    % Beispiele aus der "Code like a Pythonista" praesentation
\subsection{ALLES ist ein Objekt}
\subsection{Zen of Python} 
\begin{frame}[fragile]
    \frametitle{Zen of Python}

    \begin{exampleblock}{python}
    \begin{lstlisting}
>>> import this
    \end{lstlisting}
    \end{exampleblock}
\end{frame}
    % import this

\section{Standard Library}
%%%%%%%%%%%%%%%%%%%%%%%%%%
\subsection{Importing modules}
\subsection{sys} 
    % parameter einlesen, basic informationen auslesen
\subsection{os, os.path} 
    % path handling, 
\subsection{RTFM -- Hilfe zur Selbsthilfe}  
    %  help(), pydoc, online doku 

\section{Input/Output } 
    % auf Dateien, stdin/out/err

% Beispiel
\section{re Modul}
% Beispiel

\end{document}
