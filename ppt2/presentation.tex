\documentclass{beamer}
\usepackage[utf8]{inputenc}
\usepackage[ngerman]{babel}
\usepackage{graphicx}
\usepackage{lmodern}
\usepackage{listings}
\usepackage{color}

\lstset{
    language=Python,
    basicstyle=\scriptsize,
    breakatwhitespace=true,breaklines=true,%
    tabsize=2, % Tabulatorbreite
    showstringspaces=false
}

\hypersetup{
    pdftitle={\inserttitle},
    pdfsubject={\inserttitle},
    pdfauthor={\insertauthor},
    pdfkeywords={\inserttitle},
    pdfstartview={FitH}
}

\usetheme{Pittsburgh}
%\usetheme{Frankfurt}
\usecolortheme{whale}
%\usecolortheme{beaver}
\beamertemplatenavigationsymbolsempty
\titlegraphic{\includegraphics[height=2cm]{hk-logo.png}}
%\logo{\includegraphics[height=1cm]{hk-logo.png}}

\title[Python Workshop]{Hackinggroup -- Python Workshop -- Part 2}
\subtitle{A tale about dutch ducks with a fable for British comedy}
\author[Thomas Kastner, Michael Rodler]{Thomas Kastner\and Michael Rodler}
\date{\today}

\begin{document}

\begin{frame}
    \titlepage
\end{frame}

\section*{Introduction}
%%%%%%%%%%%%%%%%%%%%%%

\begin{frame}
    \frametitle{Introduction -- Wer san ma denn?}

    \begin{block}{Michael Rodler}
    \begin{itemize}
        \item aka f0rk, f0rki, f0rkmaster, Gabel, etc.
        \item Student SIB09
        \item 3 years coding python for fun
        \item 3 months coding python for profit
    \end{itemize}
    \end{block}

    \begin{block}{Thomas Kastner}
    \begin{itemize}
        \item aka br3z3l, tom
        \item Student SIB08
        \item 4 years coding python for fun
    \end{itemize}
    \end{block}
\end{frame}


\begin{frame}[fragile,allowframebreaks]
    \frametitle{Table of contents -- Wos ma heit mochn}
    \tableofcontents
    And because it's a so called ``Workshop'' we will also write some code together ;)
\end{frame}

\section{urllib}
%Website fetchen mit urllib

%%% \begin{beispiel} %%%

%%% \end{beispiel} %%%


\section{mechanize}
%Website crawlen mit mechanize

%%% \begin{beispiel} %%%

%%% \end{beispiel} %%%


\section{paramiko}
%Ergebnisse per SSH/SFTP auf server speicher?

%%% \begin{beispiel} %%%

%%% \end{beispiel} %%%


\section{twisted}
%Ergebnisse im IRC posten (BOTNETT!!!! :) )?

%%% \begin{beispiel} %%%

%%% \end{beispiel} %%%



\begin{frame}
	\frametitle{References}
	\begin{thebibliography}{3}
	\bibitem{kopf} the authors' epic python knowledge
	\bibitem{pydoc} \url{http://docs.python.org/}
	\end{thebibliography}
\end{frame}

\end{document}