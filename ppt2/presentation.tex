\documentclass{beamer}
\usepackage[utf8]{inputenc}
\usepackage[ngerman]{babel}
\usepackage{graphicx}
\usepackage{lmodern}
\usepackage{listings}
\usepackage{color}

\lstset{
    language=Python,
    basicstyle=\scriptsize,
    breakatwhitespace=true,breaklines=true,%
    tabsize=2, % Tabulatorbreite
    showstringspaces=false
}

\hypersetup{
    pdftitle={\inserttitle},
    pdfsubject={\inserttitle},
    pdfauthor={\insertauthor},
    pdfkeywords={\inserttitle},
    pdfstartview={FitH}
}

\usetheme{Pittsburgh}
\usecolortheme{whale}
\beamertemplatenavigationsymbolsempty
\titlegraphic{\includegraphics[height=2cm]{hk-logo.png}}

\title[Python Workshop]{Hackinggroup -- Python Workshop -- Part 2}
\subtitle{A tale about dutch ducks with a fable for British comedy}
\author[Thomas Kastner, Michael Rodler]{Thomas Kastner\and Michael Rodler}
\date{\today}

\begin{document}

\begin{frame}
    \titlepage
\end{frame}

\section*{Introduction}
%%%%%%%%%%%%%%%%%%%%%%%

\begin{frame}
    \frametitle{Introduction -- Wer san ma denn?}

    \begin{block}{Michael Rodler}
    \begin{itemize}
        \item aka f0rk, f0rki, f0rkmaster, Gabel, etc.
        \item Student SIB09
        \item 3 years coding python for fun
        \item 3 months coding python for profit
    \end{itemize}
    \end{block}

    \begin{block}{Thomas Kastner}
    \begin{itemize}
        \item aka br3z3l, tom
        \item Student SIB08
        \item 4 years coding python for fun
    \end{itemize}
    \end{block}
\end{frame}


\begin{frame}[fragile]
    \frametitle{Table of contents -- Wos ma heit mochn}
    \tableofcontents
    And because it's called a \emph{Workshop} we will also write some code together ;)
\end{frame}

\section{Some stuff we forgot}	% fork
%%%%%%%%%%%%%%%%%%%%%%%%%%%%%%

\begin{frame}[fragile]
	\frametitle{more syntactic sugar}

	\begin{exampleblock}{python}
	\begin{lstlisting}
>>> a = ""; b = "foo"
>>> x = a or b
>>> x = "a" * 21
>>> x = [1,2,3] * 5
	\end{lstlisting}
	\end{exampleblock}

	\begin{exampleblock}{python -- some operators}
	\begin{lstlisting}
>>> x = 42**2 % 1337
>>> 0xD5 & 0377 ^ 0xFF
	\end{lstlisting}
	\end{exampleblock}	
	
	\begin{exampleblock}{python -- converting types}
	\begin{lstlisting}
>>> str(42)
>>> list((1,2,3,4))
>>> int("42")	
	\end{lstlisting}
	\end{exampleblock}
\end{frame}

% os.FORK!!!!
% sys und os usen (systemnah programmieren)
% system befehle ausführen (subprocess)
% REGEX!

\section{Python performance}	% tom
%%%%%%%%%%%%%%%%%%%%%%%%%%%%

\section{List Comprehensions}	% tom
%%%%%%%%%%%%%%%%%%%%%%%%%%%%%

\begin{frame}[fragile]
	\frametitle{List Comprehension and Generator Expression}
	\begin{exampleblock}{python}
	\begin{lstlisting}
>>> n = [ i**2 for i in range(100) if i & 1 == 0 ]
>>> import os
>>> zipfiles = [ s for s in os.listdir(".") if s.endswith(".zip") ]
	\end{lstlisting}
	\end{exampleblock}
	
	\begin{exampleblock}{python}
	\begin{lstlisting}
>>> for x in ( )
>>> def f(arg):
...     do_something_with(arg)
>>> x = f( s for s in blafu )
	\end{lstlisting}
	\end{exampleblock}
\end{frame}

\section{Exceptions}	% DONE
%%%%%%%%%%%%%%%%%%%%

\begin{frame}[fragile]
	\frametitle{Exceptions -- oda wenn ois ind luft gehd}
	\begin{exampleblock}{python}
	\begin{lstlisting}
try:
    f = open("/dev/missing", "r")
except IOError, e:
    sys.stderr.write("Error: %s\n" % e.message)
    sys.exit(1)
	\end{lstlisting}
	\end{exampleblock}

\pause

	\begin{exampleblock}{python}
	\begin{lstlisting}
try:
    import threading as _threading
except ImportError:
    import dummy_threading as _threading
	\end{lstlisting}
	\end{exampleblock}
	
	It's easier to ask for forgiveness than permission (EAFP)
\end{frame}

\section{Object Orientated Programming}
%%%%%%%%%%%%%%%%%%%%%%%%%%%%%%%%%%%%%%%

% EVERYTHING IS AN OBJECT!
\begin{frame}[plain]
	\begin{center}
	\huge{\textbf{EVERYTHING IS AN OBJECT!}}
	\end{center}
\end{frame}

% Creating classes		-- fork
	%like the ability to subclass most built-in types
% Konstruktor			-- fork

% public/private		-- fork
% dynamic typing, duck typing (or why java-style interfaces suck)	-- tom
% Special Methods (Operator Overloading)							-- tom
% Inheritance		-- fork
>>> class A(object):
...     def foo(self):
...             print "A"
...
>>> class B(object):
...     def foo(self):
...             print "B"
...
>>> class C(A,B):
...     pass
...
>>> class D(B,A):
...     pass
...
>>> c = C()
>>> c.foo()
A
>>> d = D()
>>> d.foo()
B

\section{The Cheeseshop}	% fork
%%%%%%%%%%%%%%%%%%%%%%%%
% cheeseshop, easy_install, pip

\section{Advanced harvesting}
%%%%%%%%%%%%%%%%%%%%%%%%%%%%%


\subsection*{urllib}	% fork
\begin{frame}[fragile]
	\frametitle{urllib, urllib2 -- working with URLs}
\end{frame}

\subsection*{mechanize} % fork
\subsection*{twisted}	% tom
\subsection*{multithreaded / concurrent}	% tom

\section{FYI}
%%%%%%%%%%%%%

\subsection*{some low level stuff}	% fork
% linksammlung

\begin{frame}
	\frametitle{struct -- working with binary data}
	% codesnippet
\end{frame}

\begin{frame}
	\frametitle{ctypes -- accessing c functions}
	\emph{PyDbg} -- scriptable windows debugger, written in python
\end{frame}

\subsection*{ssh}
	% paramiko

\subsection*{network}
	% twisted
	% asyncore/asynchat (builtin)

\subsection*{web programming}
	% django
	% cherrypy
	% jinja2
	% ...

\subsection*{databases}
    % DB API
    % sqlalchemy
    % elixir

\begin{frame}
	\frametitle{References}
	\begin{thebibliography}{3}
	\bibitem{kopf} the authors' epic python knowledge
	\bibitem{pydoc} \url{http://docs.python.org/}
	\end{thebibliography}
\end{frame}

\end{document}