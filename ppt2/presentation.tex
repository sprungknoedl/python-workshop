\documentclass{beamer}
\usepackage[utf8]{inputenc}
\usepackage[ngerman]{babel}
\usepackage{graphicx}
\usepackage{lmodern}
\usepackage{listings}
\usepackage{xcolor}

\lstset{
    language=Python,
    basicstyle=\scriptsize,
    breakatwhitespace=true,breaklines=true,%
    tabsize=2, % Tabulatorbreite
    showstringspaces=false
}

\hypersetup{
    pdftitle={\inserttitle},
    pdfsubject={\inserttitle},
    pdfauthor={\insertauthor},
    pdfkeywords={\inserttitle},
    pdfstartview={FitH}
}

\usetheme{Pittsburgh}
\usecolortheme{whale}
\beamertemplatenavigationsymbolsempty
\titlegraphic{\includegraphics[height=2cm]{hk-logo.png}}

\title[Python Workshop]{Hackinggroup -- Python Workshop -- Part 2}
\subtitle{A tale about dutch ducks with a fable for British comedy}
\author[Thomas Kastner, Michael Rodler]{Thomas Kastner\and Michael Rodler}
\date{\today}

\begin{document}

\begin{frame}
    \titlepage
\end{frame}

\section*{Introduction}
%%%%%%%%%%%%%%%%%%%%%%%

\begin{frame}
    \frametitle{Introduction -- Wer san ma denn?}

    \begin{block}{Michael Rodler}
    \begin{itemize}
        \item aka f0rk, f0rki, f0rkmaster, Gabel, etc.
        \item Student SIB09
        \item 3 years coding python for fun
        \item 3 months coding python for profit
    \end{itemize}
    \end{block}

    \begin{block}{Thomas Kastner}
    \begin{itemize}
        \item aka br3z3l, tom
        \item Student SIB08
        \item 4 years coding python for fun
    \end{itemize}
    \end{block}
\end{frame}


\begin{frame}[fragile]
    \frametitle{Table of contents -- Wos ma heit mochn}
    \tableofcontents
    And because it's called a \emph{Workshop} we will also write some code together ;)
\end{frame}

\section{Some stuff we forgot}	% fork
%%%%%%%%%%%%%%%%%%%%%%%%%%%%%%

\begin{frame}[fragile]
	\frametitle{more syntactic sugar}

	\begin{exampleblock}{python -- syntactic sugar}
	\begin{lstlisting}
>>> a = ""; b = "foo"
>>> x = a or b
>>> x = "a" * 21
>>> x = [1,2,3] * 5
	\end{lstlisting}
	\end{exampleblock}

	\begin{exampleblock}{python -- some operators}
	\begin{lstlisting}
>>> x = 42**2 % 1337
>>> 0xD5 & 0377 ^ 0xFF
	\end{lstlisting}
	\end{exampleblock}	
	
	\begin{exampleblock}{python -- converting types}
	\begin{lstlisting}
>>> str(42)
>>> list((1,2,3,4))
>>> int("42")	
	\end{lstlisting}
	\end{exampleblock}
\end{frame}

% sys und os usen (systemnah programmieren)
% os.FORK!!!!
\begin{frame}[fragile,allowframebreaks]
	\frametitle{os -- some more stuff}
	\begin{exampleblock}{python -- os}
	\begin{lstlisting}
>>> os.getloadavg()
>>> if os.getuid() == 0:
...     os.setuid(1000)
>>> open(os.devnull, 'w').write('This is sent to nirvana')
>>> os.killpg(1337,9)
	\end{lstlisting}
	\end{exampleblock}

	\begin{exampleblock}{python -- os.walk}
	\begin{lstlisting}
from os.path import join, getsize
for root, dirs, files in os.walk("./Code"):
    print root, "consumes",
    print sum([getsize(join(root, name)) for name in files]),
    print "bytes in", len(files), "non-directory files"
	\end{lstlisting}
	\end{exampleblock}

\newpage
	\begin{exampleblock}{python -- a fork bomb}
	\begin{lstlisting}
import os
while True:
    pid = os.fork()
    if pid == 0:
        print "Hello I'm a child :D"
	\end{lstlisting}
	\end{exampleblock}
\end{frame}

% system befehle ausführen (subprocess)
\begin{frame}[fragile]
	\frametitle{subprocces -- execute processes}
	\begin{exampleblock}{python}
	\begin{lstlisting}
>>> p = subprocess.Popen(["ls", "-l", "-a"])
>>> retval = subprocess.call(["rm", "-f", "./somefile"])
	\end{lstlisting}
	\end{exampleblock}
\pause
	\begin{exampleblock}{python}
	\begin{lstlisting}
filename = input("What file would you like to display?\n")
subprocess.call("cat " + filename, shell=True)
	\end{lstlisting}
	\end{exampleblock}
\pause
	type in "non\_existent; rm -rf / \#"\\
	Oh noes, command injection... this is bad ;)
\end{frame}

% REGEX!
\begin{frame}[fragile]
	\frametitle{re -- regex again}
	\begin{exampleblock}{python}
	\begin{lstlisting}
>>> r = re.compile(r"(\w{31}=)")
>>> m = r.search("This is some random Text eS4H0NWnnFGd8cCUavc6m2DwjRUzm6h= which contains a flag ;)")
>>> if m:
...     print m.groups()
	\end{lstlisting}
	\end{exampleblock}
\pause
	\begin{exampleblock}{python}
	\begin{lstlisting}
>>> r = re.compile(r"\d{1,3}\.\d{1,3}\.\d{1,3}\.\d{1,3}")
>>> valid = r.match("10.13.37.0") is not None
>>> valid = r.match(" 10.13.37.0") is not None
>>> valid = r.search(" 10.13.37.0") is not None
	\end{lstlisting}
	\end{exampleblock}
\end{frame}


\section{Python Comparison}	% tom
%%%%%%%%%%%%%%%%%%%%%%%%%%%%
\begin{frame}[fragile,allowframebreaks]
    \frametitle{Comparison with other languages}
    \begin{block}{vs. C/C++}
    \begin{itemize}
        \item write less code with more features in less time
        \item batteries included
        \item python bottlenecks can be optimized with modules written in C/C++
    \end{itemize}
    \end{block}

    \begin{block}{vs. Java}
    \begin{itemize}
        \item faster to write code, less boilerplate code
        \item better api\footnote{\scriptsize \url{paulbuchheit.blogspot.com/2007/05/amazingly-bad-apis.html}}
        \item python 2-50x slower than java
        \item python+psyco 1-5x slower than java
        \item real oop
    \end{itemize}
    \end{block}
    
    \begin{block}{vs. PHP}
    \begin{itemize}
        \item faster than php
        \item more use cases
        \begin{itemize}
        	\item more libraries
        \end{itemize}
        \item NOT ugly
    \end{itemize}
    \end{block}

    \begin{alertblock}{conclusions}
    \begin{itemize}
        \item start of python interpreter costs! %TODO: wtf?
        %\item much faster coding
        %\begin{itemize}
        	\item rapid development %$\rightarrow$ more time to optimize bottlenecks
        %\end{itemize}
        \item \textbf{most performance bottlenecks are wrong algorithms}
        \item \url{http://wiki.python.org/moin/PythonSpeed}
        \begin{itemize}
        	\item new Interpreters: PyPy, Unladen Swallow, etc. are faster
        \end{itemize}
    \end{itemize}
    \end{alertblock}
\end{frame}

\section{List Comprehensions}	% tom
%%%%%%%%%%%%%%%%%%%%%%%%%%%%%

\begin{frame}[fragile]
	\frametitle{List Comprehension and Generator Expression}
	\begin{exampleblock}{python -- list comprehensions}
	\begin{lstlisting}
>>> n = [i for i in range(100) if i % 2]
>>> type(n)
<type 'list'>
>>> import os
>>> zipfiles = [n for r, _, _ in os.walk(".") if n.endswith(".zip")]
	\end{lstlisting}
	\end{exampleblock}
	
	\begin{itemize}
		\item quickly construct lists
		\item list ist constructed and then returned
	\end{itemize}

\end{frame}

\begin{frame}[fragile]
	\frametitle{List Comprehension and Generator Expression}
	\begin{exampleblock}{python -- generator expressions}
	\begin{lstlisting}
>>> n = (i for i in xrange(100) if i % 2)
>>> type(n)
<type 'generator'>
>>> import os
>>> zipfiles = (n for n, _, _ in os.walk(".") if n.endswith(".zip"))
	\end{lstlisting}
	\end{exampleblock}

	\begin{itemize}
		\item only one item is accessible
		\item therfore consumes less memory
		\item computed on the fly
	\end{itemize}
\end{frame}

\section{Exceptions}	% DONE
%%%%%%%%%%%%%%%%%%%%

\begin{frame}[fragile]
	\frametitle{Exceptions -- oda wenn ois ind luft gehd}
	\begin{exampleblock}{python}
	\begin{lstlisting}
try:
    f = open("/dev/missing", "r")
except IOError, e:
    sys.stderr.write("Error: %s\n" % e.message)
    sys.exit(1)
	\end{lstlisting}
	\end{exampleblock}

\pause

	\begin{exampleblock}{python}
	\begin{lstlisting}
try:
    import threading as _threading
except ImportError:
    import dummy_threading as _threading
	\end{lstlisting}
	\end{exampleblock}
	
	It's easier to ask for forgiveness than permission (EAFP)
\end{frame}

\section{Object Oriented Programming}
%%%%%%%%%%%%%%%%%%%%%%%%%%%%%%%%%%%%%%%

% EVERYTHING IS AN OBJECT!
\begin{frame}[plain]
	\begin{center}
	\huge{\textbf{EVERYTHING IS AN OBJECT!}}
	\end{center}
\end{frame}

\begin{frame}[fragile]
	\frametitle{Everything is an Object}
	\begin{exampleblock}{python}
	\begin{lstlisting}
>>> x = 42; y = 42
>>> x == y
True
>>> x is y
True
>>> x = []; y = []
>>> x == y
True
>>> x is y
False
	\end{lstlisting}
	\end{exampleblock}
	Numbers are actually singletons\\
	\begin{itemize}
		\item \emph{is} checks for object indentity
		\item \emph{=} checks for object equality
	\end{itemize}
\end{frame}

% Creating classes		-- fork
	% new-style --> like the ability to subclass most built-in types
% Konstruktor			-- fork
\begin{frame}[fragile]
	\frametitle{Class definition}
	\begin{exampleblock}{python}
	\begin{lstlisting}
>>> class MyClass(object):
...     def __init__(self, y):
...         self.x = 42
...         self.y = y
...     def func(self):
...         return self.x
... 
>>> c = MyClass(21)
>>> print c.func()
42
>>> print c.y
21
	\end{lstlisting}
	\end{exampleblock}
	
	explicit inheritance from \emph{object} is needed to specify ''new-style'' classes
\end{frame}

% public/private		-- fork
\begin{frame}[fragile]
	\frametitle{public/private attributes}
	everything is public.\\
	private attributes/methods are written with preceding underscore
	\begin{exampleblock}{python -- quasi private attribute}
	\begin{lstlisting}
>>> class Fu(object):
...     def __init__(self):
...         self._x = 42
...         self.__y = 21
>>> f = Fu()
>>> f._x
42
>>> f.__y
Traceback (most recent call last):
  File "<input>", line 1, in <module>
AttributeError: 'Fu' object has no attribute '__y'
>>> f.__dict__
{'_Fu__y': 21, '_x': 42}
>>> f._Fu__y
21
	\end{lstlisting}
	\end{exampleblock}
\end{frame}


\begin{frame}[fragile, allowframebreaks]
	\frametitle{quasi private from a heir's viewpoint}
	\begin{exampleblock}{python}
	\begin{lstlisting}
>>> class Fu(object):
...     def __init__(self):
...         self._x = 42
...         self.__y = 21
...     def barf(self):
...         return self.__y
>>> class Bla(Fu):
...     pass
>>> b = Bla()
>>> b.barf()
21
>>> b.__y = 23
>>> b.barf()
21
>>> print b.__dict__
{'__y': 23, '_Fu__y': 21, '_x': 42}
	\end{lstlisting}
	\end{exampleblock}
\framebreak
	\begin{exampleblock}{python}
	\begin{lstlisting}
>>> class Bla(Fu):
...     def nom(self):
...         self.__y = 34
>>> b = Bla()
>>> b.barf()
21
>>> b.nom()
>>> b.barf()
21
>>> print b.__dict__
{'_Bla__y': 34, '_Fu__y': 21, '_x': 42}
	\end{lstlisting}
	\end{exampleblock}
\end{frame}

% Inheritance		-- fork
\begin{frame}[fragile]
	\frametitle{Inheritance}
	\begin{exampleblock}{python -- multiple inheritance}
	\begin{lstlisting}
>>> class A(object):
...     def foo(self):
...             print "A"
...
>>> class B(object):
...     def foo(self):
...             print "B"
...
>>> class C(A,B):
...     pass
...
>>> class D(B,A):
...     pass
...
    \end{lstlisting}
\pause
	\begin{lstlisting}
>>> c = C()
>>> c.foo()
A
>>> d = D()
>>> d.foo()
B	
	\end{lstlisting}
	\end{exampleblock}
\end{frame}

% dynamic typing, duck typing (or why java-style interfaces suck)	-- tom
% Special Methods (Operator Overloading)							-- tom


\section{The Cheeseshop}	% fork
%%%%%%%%%%%%%%%%%%%%%%%%
% cheeseshop, easy_install, pip

\begin{frame}[fragile]
	\frametitle{The cheeseshop -- installing stuff}
	Python Package Index\\
	\url{http://pypi.python.org}
	\begin{exampleblock}{python}
	\begin{lstlisting}[language=bash]
[py@workshop ~]$ easy_install pelican
[py@workshop ~]$ easy_install pip
[py@workshop ~]$ pip uninstall pelican
	\end{lstlisting}
	\end{exampleblock}
	
	Will install python ''.egg''s
\end{frame}

\section{Advanced harvesting}
%%%%%%%%%%%%%%%%%%%%%%%%%%%%%


\subsection*{urllib}	% fork
\begin{frame}[fragile]
	\frametitle{urllib, urllib2 -- working with URLs}
	\begin{exampleblock}{python -- urlopen}
	\begin{lstlisting}
>>> site = urllib.urlopen("http://f0rki.at")
>>> print f.read()
	\end{lstlisting}
	\end{exampleblock}

\pause

	\begin{exampleblock}{python -- url encoding}
	\begin{lstlisting}
>>> import urllib
>>> params = urllib.urlencode({'spam': 1, 'eggs': 2, 'bacon': 0})
>>> f = urllib.urlopen("http://www.example.com/cgi-bin/query", params)
>>> print f.read()
	\end{lstlisting}
	\end{exampleblock}
\end{frame}

\subsection*{mechanize} % fork
\begin{frame}[fragile]
	\frametitle{mechanize -- scripted browser}
	behaves like a normal browser (with cookies and stuff)
	\begin{exampleblock}{python}
	\begin{lstlisting}
br = mechanize.Browser()
br.open("http://www.example.com/")
response1 = br.follow_link(text_regex=r"cheese\s*shop", nr=1)
assert br.viewing_html()
print br.title()
print response1.geturl()
print response1.info()
print response1.read()
br.select_form(name="order")
br["cheeses"] = ["mozzarella", "caerphilly"]
response2 = br.submit()
print br.form
	\end{lstlisting}
	\end{exampleblock}
\end{frame}

\subsection*{twisted}	% tom
\begin{frame}[fragile]
	\frametitle{twisted -- building the engine of your internet}
	\begin{block}{Protocol}
	\begin{itemize}
		\item a
	\end{itemize}
	\end{block}

	\begin{block}{Factory}
	\begin{itemize}
		\item a
	\end{itemize}
	\end{block}

	\begin{block}{Reactor}
	\begin{itemize}
		\item a
	\end{itemize}
	\end{block}
\end{frame}

\subsection*{multithreaded / concurrent}	% tom

% mention multiprocessing
% include the following
% http://wiki.python.org/moin/GlobalInterpreterLock
% http://wiki.python.org/moin/StacklessPython

\section{FYI}
%%%%%%%%%%%%%

\subsection*{some low level stuff}	% fork
% linksammlung

\begin{frame}[fragile]
	\frametitle{struct -- working with binary data}
	\begin{exampleblock}{python -- unpacking ip header}
	\begin{lstlisting}
>>> pkt = "E\x00\x00\x9c\x00\x00@\x00@\x11\xbe-\n\r%\x02\n\r%\x03"
>>> iphdr = struct.unpack("!BBHHHBBHII", pkt[0:20])
>>> print iphdr
(69, 0, 156, 0, 16384, 64, 17, 48685, 168633602, 168633603)
	\end{lstlisting}
	\end{exampleblock}
	'!' $\rightarrow$ Network Byte-Order aka. Big-Endian\\[0.4cm]
	\begin{tabular}{|l|l|}	
	Format & C type\\
	\hline
	B & unsigned char\\
	H & unsigned short\\
	I & unsgined int\\	
	\end{tabular}
\end{frame}

\begin{frame}[fragile]
	\frametitle{ctypes -- accessing c functions}
	\begin{exampleblock}{python -- calling native printf}
	\begin{lstlisting}
from ctypes import *
libc = CDLL("libc.so.6")
x = libc.time(None)
y = libc.printf("Test: %d, %f\n", 42, c_double(13.37))
print "c printf printed", y, "characters at", x
	\end{lstlisting}
	\end{exampleblock}
\pause
	\begin{itemize}
		\item Quick C Interaction
		\begin{itemize}
			\item speed-ups
			\item Accessing C library without bindings
		\end{itemize}
	\end{itemize}
	Example:\\
	\emph{PyDbg} -- open-source scriptable windows debugger, written in python using ctypes
\end{frame}

\subsection*{ssh}
	% paramiko
\begin{frame}
	\frametitle{ssh -- paramiko}
	\begin{block}{paramiko}
	\begin{itemize}
		\item \url{http://www.lag.net/paramiko/}
		\item many features
		\begin{itemize}
			\item shell/command execution
			\item Agent
			\item SFTP Support
		\end{itemize}
		\item both low and high level access
	\end{itemize}
	\end{block}
\end{frame}

\subsection*{network}
	% twisted
	% asyncore/asynchat (builtin)
\begin{frame}[allowframebreaks]
	\frametitle{network programming}
	\begin{block}{twisted}
	\begin{itemize}
		\item \url{http://twistedmatrix.com/}
		\item event-driven networking engine
		\item Implements a large number of protocols
		\item very good framework
		\begin{itemize}
			\item i.e. you'll have to do things their way
		\end{itemize}
	\end{itemize}
	\end{block}	
\framebreak
	\begin{block}{asyncore}
	\begin{itemize}
		\item builtin \url{http://docs.python.org/library/asyncore.html}
		\item low-level socket handling
	\end{itemize}
	\end{block}
	
	\begin{block}{asynchat}
	\begin{itemize}
		\item builtin \url{http://docs.python.org/library/asynchat.html}
		\item uses asyncore
		\item for protocols, with string terminated elements
	\end{itemize}
	\end{block}
\end{frame}

\subsection*{web programming}
	% django
	% cherrypy
	% jinja2
	% ...
\begin{frame}[allowframebreaks]
	\frametitle{Web programming}
	\begin{block}{Django}
	\begin{itemize}
		\item \url{http://www.djangoproject.com/}
		\item ''The Web Framework for perfectionists with deadlines.''
		\item Model-View-Controller
		\item Database-Driven
	\end{itemize}
	\end{block}
	
	\begin{block}{CherryPy}
	\begin{itemize}
		\item \url{http://www.cherrypy.org/}
		\item handles only HTTP
		\item more flexible
	\end{itemize}
	\end{block}
\framebreak
	\begin{block}{Jinja}
	\begin{itemize}
		\item \url{http://jinja.pocoo.org/}
		\item mighty (html) templating engine
	\end{itemize}
	\end{block}
	
	\begin{block}{zope}
	\begin{itemize}
		\item \url{http://www.zope.org/}
		\item Web application server
		\item recognized as ''python killer app'' 
	\end{itemize}
	\end{block}
\end{frame}

\subsection*{databases}
    % DB API
    % sqlalchemy
    % elixir
\begin{frame}
	\frametitle{Working with databases}
	\begin{block}{Python DB API 2.0}
	\begin{itemize}
		\item Standard ''Interface'' for Database modules
		\item \url{http://www.python.org/dev/peps/pep-0249/}
	\end{itemize}
	\end{block}
	
	\begin{block}{sqlalchemy}
	\begin{itemize}
		\item \url{http://www.sqlalchemy.org/}
		\item object-relational mapper
		\begin{itemize}
			\item \url{http://elixir.ematia.de}
			\item declarative extension
		\end{itemize}
	\end{itemize}
	\end{block}
\end{frame}

\subsection*{Python development}

\begin{frame}[allowframebreaks]
	\frametitle{Usefull stuff}
	
	\begin{block}{Alternative interpreters}
	\begin{itemize}
		\item IPython
		\item BPython
		\item \url{http://wiki.python.org/moin/PythonEditors\#EnhancedPythonshells}
	\end{itemize}
	\end{block}

	\begin{block}{editors}
	\begin{itemize}
		\item vim
		\item scribes
		\item \url{http://wiki.python.org/moin/PythonEditors}
	\end{itemize}
	\end{block}

\framebreak

	\begin{block}{debuggers}
	\begin{itemize}
		\item pdb
		\item winpdb (rpdb2)
		\item pydb
		\item \url{http://wiki.python.org/moin/PythonDebuggers}
	\end{itemize}
	\end{block}
	
	\begin{block}{Integrated Development Environment}
	\begin{itemize}
		\item Eclipse with Pydev
		\item NetBeans
		\item \url{http://wiki.python.org/moin/IntegratedDevelopmentEnvironments}
	\end{itemize}
	\end{block}
\end{frame}

\begin{frame}
	\frametitle{References}
	\begin{thebibliography}{3}
	\bibitem{kopf} the authors' epic python knowledge
	\bibitem{pydoc} \url{http://docs.python.org/}
	\end{thebibliography}
\end{frame}

\end{document}
